\documentclass[11pt]{report}

\usepackage{geometry}
\geometry{a4paper}
\usepackage{graphicx}
\graphicspath{ {./images/} }
\DeclareGraphicsExtensions{.png,.pdf,.jpg}
\usepackage{array}
\usepackage{multirow}
\usepackage{mathtools}
\usepackage[T1]{fontenc}
%\usepackage[latin9]{inputenc}
\usepackage{babel}
\usepackage[table]{xcolor}
\usepackage{collcell}
\usepackage{hhline}
\usepackage{pgf}


\begin{document}

\noindent
\textbf{Javier Ricardo Becerra Bedoya}

\noindent
\textbf{Pontificia Universidad Javeriana, Bogotá}

\noindent
\textbf{Trabajo de grado: Clasificación de actividad humana con acelerómetros de Smartphones.}

Bogotá,   \indent   \indent de  Septiembre del 2017

\medskip
\noindent
\textbf{Introducción y propósito}

Yo, Javier Ricardo Becerra Bedoya, estudiante de ingeniería electrónica de la Pontificia Universidad Javeriana de Bogotá, en la realización del trabajo final del pregrado, he solicitado su colaboración y participación voluntaria, en la recolección de datos de movimiento mediante un Smartphone ubicado en dos lugares distintos del cuerpo (cintura y brazo). Lo anterior, con el objetivo de realizar un sistema  de clasificación de actividad humana automático, que aportará a la investigación en el monitoreo de enfermedades que afectan la capacidad motora, como el Párkinson, para buscar una mejora en la calidad de vida de estos pacientes.
\par
\medskip
\noindent
Mediante este documento, yo, garantizo que no será vulnerada su integridad física ni mental y que sus datos serán tratados confidencialmente, teniendo en cuenta que este estudio se cataloga como investigación de riesgo mínimo según la RESOLUCIÓN No 008430 DE 1993 del Ministerio de Salud de Colombia. 

\par
\medskip
\noindent
\textbf{Procedimiento a realizar} 

\par
\medskip
\noindent
Se le entregará al voluntario un Smartphone Samsung Galaxy SII con dos accesorios para localizarlo en dos partes distintas del cuerpo (una riñonera y banda para el brazo). Una vez haya sido localizado en una de las dos partes del cuerpo se procederá a realizar el protocolo de pruebas indicado en la tabla con los tiempos designados, grabando en tiempo real los datos recolectados por el acelerómetro y giroscopio del celular.  Se le indicará al voluntario el recorrido exacto para cada actividad y los tiempos en los que debe ir cambiando de la una a la otra. El procedimiento debe ser repetido para la otra posición del Smartphone.



\begin{table}[h!]
\begin{center}
\begin{tabular}{ |c|c|c|c|c|c| } 
\hline
\textbf{No.} & \textbf{Estáticas} & \textbf{Tiempo (S)} & \textbf{No.} & \textbf{Dinámicas} & \textbf{Tiempo (S)} \\
\hline
0 & Comienzo (de pie) & 0 & 7 & Caminar (1) & 15 \\ 
1&  Reposo (De pie) & 15 & 8 & Caminar(2) & 15\\ 
2&  Sentar & 15 & 9 & Bajar Escaleras (1) & 36\\ 
3&  De Pie & 15 & 10 & Subir Escaleras (2) & 36\\ 
4&  Acostarse & 15 & 11 &  & \\ 
5&  Sentar & 15 & 12 &  & \\ 
6&  Acostarse & 15 & 13 &  & \\ 
\hline
\hline
&   &  &  & \textbf{Total} & \textbf{192}\\ 
\hline
\end{tabular}
\caption{Protocolo de actividades a realizar}
\end{center}
\end{table}

\par
\medskip
\noindent
\textbf{Consentimiento informado}

\par
\medskip
\noindent
Yo,  \indent   \indent  \indent   \indent \indent   \indent  \indent   \indent  \indent   \indent  \indent   \indent he leído y escuchado los propósitos, objetivos y procedimientos a realizar en el presente estudio, he tenido la oportunidad de preguntar sobre este y me ha sido brindada una respuesta óptima. Decido participar voluntariamente en la toma de los datos necesarios y tengo claridad de que mis datos serán tratados de manera confidencial y con fines académicos. Además, soy consciente de que me puedo retirar del estudio en el momento que lo desee y que el procedimiento planteado no afecta mi integridad física ni mental. 

\par
\medskip
\medskip
\medskip
\noindent
\textbf{Firma: }

\par
\medskip
\noindent
\textbf{Fecha: } 

\par
\medskip
\noindent
\textbf{Documento de identidad: } 



\end{document}